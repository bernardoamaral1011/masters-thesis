% #############################################################################
% Abstract Text
% !TEX root = ../main.tex
% #############################################################################
% use \noindent in first paragraph
\noindent 
Air pollution is a global problem due to its consequences on population health. World Health Organization estimates that every year 7 million deaths are caused by air pollution related to small inhalable particles. This work intends to provide useful research in the scope of air pollution monitoring and spatial prediction, through the development of a pioneer Narrow-band Internet of Things (NB-IoT) based system for the measurement of particulate matter with a diameter smaller than 10 micrometers, using the low-cost PMS5003 sensor. This system integration in official monitoring clusters was tested, through the comparison with the Portuguese Environment Agency reference sensors in Lisbon, aiming to improve the spatial density of the current network. Furthermore, to enhance the resolution of available data, several algorithms, such as Ordinary Kriging, Inverse Distance Weighting, Linear Interpolation, Nearest Neighbors and Fuzzy Boolean Nets were tested for spatial interpolation through cross-validation, with data from 2013 to 2017. Finally, a web platform for the visualization of live interpolated PM10 concentration was produced. NB-IoT presented low coverage in the city, and comparisons between sensor measurements showed several inconsistencies in the low-cost sensor. Ordinary Kriging and Inverse Distance Weighting, with a zero power function, presented the best performance, indicating a low distance correlation between the reference stations. The developed application allowed a successful representation of high resolution interpolated data. Currently, the NB-IoT technology is not well established in Portugal and further research is needed to improve the integration of low-cost solutions in air quality monitoring networks.


