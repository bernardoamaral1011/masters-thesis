% #############################################################################
% RESUMO em Português
% !TEX root = ../main.tex
% #############################################################################
% use \noindent in first paragraph
\noindent A poluição do ar é um problema global devido às suas consequências para a saúde da população. A Organização Mundial da Saúde estima que todos os anos sete milhões de mortes são causadas pela poluição do ar. Esta dissertação pretende obter informações cientificamente úteis no âmbito da monitorização e previsão espacial de poluentes ambientais através da construção de um sistema Narrow-band Internet of Things (NB-IoT) para a medida de partículas finas com um diâmetro menor que 10 micrómetros (PM10), utilizando o sensor PMS5003. A integração deste sistema em redes de monitorização oficiais foi testada, através da comparação com sensores de referência da Agência Portuguesa do Ambiente, de modo a melhorar a densidade espacial destas.
Para aumentar a resolução dos dados disponíveis, algoritmos como Ordinary Kriging, Inverse Distance Weighting, Linear Interpolation, Nearest Neighbors e Fuzzy Boolean Nets foram testados para interpolação espacial, com dados desde 2013 a 2017. Finalmente, uma plataforma web para a visualização de concentrações de PM10 em tempo real foi desenvolvida.
A tencnologia NB-IoT apresentou pouca cobertura em Lisboa, e comparações entre as medidas dos sensores mostraram inconsistências no sensor PMS5003. Ordinary Kriging e Inverse Distance Weight, com função de potência igual a zero, apresentaram o melhor desempenho, indicando pouca correlação entre as estações em termos de distância. A aplicação desenvolvida permitiu a representação de dados com alta definição. Atualmente, a tecnologia NB-IoT ainda não está estabelecida em Portugal e um maior estudo das soluções de baixo preço para a monitorização da qualidade do ar é necessário.
